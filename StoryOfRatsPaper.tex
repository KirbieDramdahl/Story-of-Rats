% Kirbie Dramdahl Honors Capstone Paper 2015

\documentclass[12pt]{article}

\setlength{\oddsidemargin}{0in}
\setlength{\evensidemargin}{0in}
\setlength{\topmargin}{0in}
\setlength{\headheight}{0in}
\setlength{\headsep}{0in}
\setlength{\textwidth}{6in}
\setlength{\textheight}{9in}
\setlength{\parindent}{0in} 

\usepackage{parskip}
\usepackage{times} %For typeface
\usepackage{graphicx}
\usepackage{algorithm}
\usepackage{algorithm,algorithmic}
\usepackage[justification=centering]{caption}[2007/12/23]
\usepackage{url}
\sloppy

\usepackage{float}
\newfloat{Query}{tbp}{lop}

\newcommand{\inset}[1]{$\in \{ {#1} \}$}

\newcommand{\citep}[1]{\cite{#1}}
\newcommand{\PPLR}[1]{$\eta_M$}
\newcommand{\LLR}[1]{$\eta_L$}

\DeclareGraphicsRule{.tif}{png}{.png}{`convert #1 `dirname #1`/`basename #1 .tif`.png}

\title{Story of Rats:\\
       The Co-Evolution of Rats and Humans}

\author{
 		M. Kirbie Dramdahl\\
        University of Minnesota, Morris\\
        Morris, MN 56267\\
        dramd002@morris.umn.edu\\
}
\date{} 

\begin{document}
\pagestyle{plain}

\maketitle

\begin{abstract}

Humans are currently the most successful species on the planet. We have permeated the globe as a result of our skill at adapting to our environment. But for as long as there have been humans, there have also been rats. And where we have gone, they have followed, with nearly equal success. It is inevitable, therefore, that the lives and histories of rats and humans have become intertwined in complex ways.

Throughout our mutual histories, humans and rats have interacted and affected one another in complex ways. Rats have latched onto humans as providers of food and shelter - whether intentionally or not - while humans have in turn viewed rats as a food source. Rats spread the disease which caused the deaths of millions of humans during the years of the Black Death, while today humans sacrifice the minds and bodies of millions of rats in order to understand their own. Rats are treated as filthy vermin in some contexts, and cherished pets in others. And despite sharing many physical and psychological characteristics with humans, rats are often feared and demonized in human culture.

The purpose of this project is to document and examine the relationship between rats and humans, and how this relationship has affected both species. How are rats and humans similar? How are they different? Has the presence of rats significantly impacted the development of humans as a species, and if so, how? Is the reverse true? How have rats impacted human culture or, more accurately, cultures? And again, is the reverse true?

Additionally, this project should ideally address questions of morality in this relationship. Are humans justified in using rats in research and other laboratory environments? Are rats entitled to rights? And what, if anything, makes humans superior to rats?

\end{abstract}

\section{Introduction} \label{Introduction}

\section{Biology} \label{Biology}

\subsection{Definition of Rat} \label{Rat Definition}

\subsection{Genetic Comparison} \label{Genetic Comparison}

\section{History} \label{History}

\section{Connections} \label{Connections}

\subsection{Disease} \label{Disease}

\subsection{Research} \label{Research}

\subsection{Companionship} \label{Companionship}

\subsection{Service} \label{Service}

\section{Conclusions} \label{Conclusions}

\section{Acknowledgements}

Thanks to Leslie Meek, Rebecca Dean, and Heather Waye.

\pagebreak

\bibliographystyle{acm}
\bibliography{StoryOfRatsBibliography}

\end{document}